\documentclass{article}
\setcounter{section}{1}

\usepackage{amssymb}
\usepackage{amsmath}
\usepackage[parfill]{parskip}

\begin{document}
\title{Sheldon Axler - Linear Algebra Done Right, 3rd edition - Chapter 2}
\author{Andrew Lim}

\def \problem#1{\subsubsection*{Problem #1}}
\def \real{\mathbf{R}}
\def \complex{\mathbf{C}}
\def \field{\mathbf{F}}

Notations are the same as what Axler uses. For example, we use boldface $\real$
for the reals, and $\field$ is $\real$ or $\complex$, not a general field.

\section{Finite-Dimensional Vector Spaces}

\subsection{Span and Linear Independence}

\problem{1}

Any $v \in V$ can be written as $a_1v_1 + a_2v_2 + a_3v_3 + a_4v_4$ for some
$a_i$. This in turn can be written as $a_1(v_1 - v_2) + (a_1 + a_2)(v_2 - v_3) +
(a_1 + a_2 + a_3)(v_3 - v_4) + (a_1 + a_2 + a_3 + a_4)v_4$. So this list also
spans $V$.

\problem{2a}

$\Rightarrow$: If $v = 0$, any nonzero choice of $a$ would satisfy $av = 0$,
violating linear independence. Thus $v \neq 0$.

$\Leftarrow$: $a = 0$ satifies $av = 0$. But $a \neq 0$ cannot, since we could
multiply both sides by $1/a$ to get $v = (1/a)0 = 0$, violating $v \neq 0$.

\problem{2b}

$\Rightarrow$: Call the vectors $v_1, v_2$. If one were a scalar multiple of the
other, say $v_2 = av_1$, then $(-a)v_1 + v_2 = 0$, violating linear
independence.

$\Leftarrow$: If the vectors were not linearly independent, then we could pick
$a_1v_1 + a_2v_2 = 0$, which we can rearrange to $v_2 = (-a_1/a_2)v_1$,
violating our assumption of no scalar multiples.

\problem{2c}

Any linear combination of these vectors $(a_1, a_2, a_3)$ is equal to $(a_1,
a_2, a_3, 0)$, which is only 0 if $a_1 = a_2 = a_3 = 0$. Thus, these are
linearly independent.

\problem{2d}

Any linear combination of these polynomials $(a_0, \ldots, a_m)$ is equal to
$a_0 + \ldots + a_mz^m$, which can only equal the additive identity polynomial
$0$ if each $a_i = 0$.

\problem{3}

A linear combination $a_1, a_2, a_3$ of these vectors equals $(3a_1 + 2a_2 +
5a_3, a_1 - 3a_2 + 9a_3, 4a_1 + 5a_2 + ta_3)$. For linear dependence, we want
there to be many solutions where this equals $(0,0,0)$.

The first two constraints give us $a_2 = 2a_3$. Substituting that into the sum
of the first two constraints, we get $4a_1 + 12a_3 = 0$; substituting into the
third constraint, we get $4a_1 + (10+t)a_3 = 0$. So $t = 2$ gives us many
possible solutions, since that makes these two constraints redundant, and
anything satisfying $a_2 = 2a_3$ and $a_1 = -3a_3$ will result in a linear
combination equaling $(0,0,0)$. For example, $a_1 = -3, a_2 = 2, a_3 = 1$ would
then result in $(0,0,0)$.

\problem{4}

$\Rightarrow$: Similar approach to problem 3. The constraints from the first two
elements, when added, give us $a_1 = -2a_3$. Substituting this into one of the
first two constraints gives $a_2 + 3a_3 = 0$; substituting this into the third
constraint gives us $2a_2 + (c-2)a_3 = 0$. These are identical for nonzero $a_i$
only if $c = 8$, so this is the only value for $c$ that can satisfy linear
dependence.

$\Leftarrow$: As per above, if $c = 8$, then any $a_1 = -2a_3$, $a_2 = -3a_3$
will create a linear combination equal to 0. For example, $(-2,-3,1)$ works.

\problem{5a}

Any linear combination equals $(a_1 + a_2) + (a_1 - a_2)i$. For this to equal 0,
$a_1 + a_2 = 0$ and $a_1 - a_2 = 0$, which is only possible if $a_1 = a_2 = 0$.

\problem{5b}

Any linear combination equals $(a_1 + a_2 - b_1 + b_2) + (a_1 - a_2 + b_1 +
b_2)i$. For this to equal 0, both of these groupings need to equal 0. Adding the
two constraints together yields $a_1 = -b_2$; substituting that in yields $a_2 =
b_1$. So any linear combination satisfying this would yield 0. For example,
$(1+i, 1-i)$ would work: $(1+i)^2 + (1-i)^2 = 0$.

\problem{6}

If they were not independent, there would exist $a_i$ not all zero such that
$a_1(v_1 - v_2) + a_2(v_2 - v_3) + a_3(v_3 - v_4) + a_4v_4 = 0$. Rearranging, we
get $a_1v_1 + (a_2 - a_1)v_2 + (a_3 - a_2)v_3 + (a_4 - a_3)v_4 = 0$.

Based on the first term, if $a_1 \neq 0$, we have a violation of the linear
independence of the $v_i$, so it must be that $a_1 = 0$. Moving to the next
term, for similar reasons, $a_2 = 0$, and then $a_3 = 0$. But then by the last
term we have $a_4 = 0$, which now violates the assumption that the $a_i$ are not
all zero. Hence, these vectors are independent.

\problem{7}

Suppose these were not linearly independent. Then there would exist $a_i$ not
all 0 such that $a_1(5v_1 - 4v_2) + a_2v_2 + \ldots + a_mv_m = 0$. Rearranging,
we get $5a_1v_1 + (a_2 - 4a_1)v_2 + \ldots + a_mv_m = 0$. By linear independence
of the $v_i$, it must be that $a_1 = 0$, then likewise $a_2 = 0$, and so on,
until we arrive at a contradiction because all the $a_i$ must be 0. Hence, these
vectors are also linearly independent.

\problem{8}

If these were not linearly independent, then some $a_i$ not all 0 would make
the sum of $\lambda a_iv_i$ equal 0. But then $\lambda a_i$ would also not all
be 0, violating the linear independence of the $v_i$. So the $\lambda v_i$ are
also linearly independent.

\problem{9}

Set $w_i = -v_i$. As proved in problem 8, if $v_i$ is linearly independent then
$w_i$ is as well, so any linearly independent $v_i$ will also gives us $w_i$
satisying the conditions of the statement. But $v_i + w_i = 0$ for all $i$, so
the sum is a list of 0s, which are not linearly independent.

\problem{10}

If $v_i + w$ are LD, then there exist $a_i$ not all 0 such that $\sum a_i(v_i +
w) = 0$. Rewrite this as $\sum a_iv_i + (\sum a_i)w = 0$. By LI of $v_i$, we
know that the first term cannot equal 0, and so likewise the second term cannot
equal 0, and therefore $\sum a_i \neq 0$ (if $w = 0$, then $v_i + w$ would
clearly not be LD). Thus we can move the second term to the right and divide by
$\sum a_i$ to get $w = \sum -(a_i / \sum a_i) v_i$, showing that $w \in
\text{span}(\{v_i\})$.

\problem{11}

$\Rightarrow$: If $w \in \text{span}(\{v_i\})$, then $\sum a_iv_i = w$ for some
$a_i$. Then $\sum a_iv_i - w = 0$. This violates the LI assumption, since we
have a linear combination of the $v_i$ and $w$ where at least one coefficient is
not zero (-1 on $w$). Therefore, $w \notin \text{span}(\{v_i\})$.

$\Leftarrow$: If $v_i$ and $w$ were not LI, then there would exist $a_i$ not all
0 such that $\sum a_iv_i + a_{m+1}w = 0$. For such $a_i$, it would not be
possible for $a_{m+1} = 0$, as then we would have $\sum a_iv_i = 0$ for $a_i$
not all 0, violating the assumption that the $v_i$ are LI. Thus, we could
rearrange to get $w = \sum (a_i/a_{m+1})v_i$, violating the assumption that $w
\notin \text{span}(\{v_i\})$. Therefore, $w \in \text{span}(\{v_i\})$.

\problem{12}

$\mathcal{P}_4(\field)$ is spanned by $(1, z, z^2, z^3, z^4)$. By 2.23, every LI
list of vectors can have at most 5 elements.

\problem{13}

In $\mathcal{P}_4(\field)$, $(1, z, z^2, z^3, z^4)$ are LI - to get any linear
combination of them to equal 0, you must multiply each of them by 0 itself.
By 2.23, a spanning list must then have length of at least 5.

\problem{14}

$\Rightarrow$: If $V$ is infinite-dimensional, no list spans $V$. We can
construct a required such sequence by starting with the empty list, which is LI,
then for $i = 1, 2, 3, \ldots$, picking $v_i$ such that it is not in the span of
the vectors before it, which is always possible since none of these lists can
span $V$.

$\Leftarrow$: Suppose $V$ were finite-dimensional. Then some list of vectors
spans it, and by 2.23, every LI list in $V$ must have length less than or equal
to the spanning list. But this is a contradiction, since we can create an LI
list of arbitrary length from our infinite sequence. Thus, $V$ is
infinite-dimensional.

\problem{15}

Consider the infinite sequence $\{v_i\}$, where $v_i$ has a 1 as its $i$th
element and 0 everywhere else. $v_1, \ldots, v_m$ is LI for any positive integer
$m$, since we cannot get the 0 sequence without multiplying each $v_i$ by 0. By
problem 14, $\field^\infty$ is infinite-dimensional.

\problem{16}

The infinite sequence of polynomials $1, z, z^2, \ldots$ defined over $[0, 1]$
is LI over its first $m$ elements for any positive integer $m$. By problem 14,
this vector space is infinite-dimensional.

\problem{17}

Each polynomial $p_j(z)$ can be written as $(z-2)(a_{j0} + \ldots +
a_{j(m-1)}z^{m-1})$ for some $a_{j0}, \ldots, a_{j(m-1)}$. Denote this
representation as $p_j(z) = (z-2)q_j(z)$.

To show that the $p_j$ are LD, we need to show that there exist $b_j$ not all 0
such that $\sum b_jp_j = 0$. This can be rewritten as $(z-2) \sum b_jq_j = 0$,
which implies that $\sum b_jq_j = 0$.

The $q_j$ are vectors in $\mathcal{P}_{m-1}(\field)$, which is itself a vector
space spanned by the list of $m$ polynomials $(1, z, \ldots, z^{m-1})$. By 2.23,
the $q_j$ must be LD, because they are a list of length $m+1$. Therefore, some
$b_j$ not all 0 exist such that $\sum b_jq_j = 0$, and thus also $\sum b_jp_j =
0$, making the $p_j$ LD.

\subsection{Bases}

\problem{1}

Any vector space with a basis that contains a vector $v$ also has a basis that
has $-v$ in place of $v$. ($v \neq 0$, since 0 cannot be in a basis, as it
allows for many linear combinations that equal 0.) Thus, the only vector space
with exactly one basis is one whose basis is the empty list, which is ${0}$.

\problem{2a}

Pretty self-evident; any $v \in \field^n$ can be written as a linear combination
of the standard basis with $a_i$ multiplying the basis vector with a 1 in the
$i$th element, and this is the only way it can be written, as only that basis
vector can affect the value of the $i$the element.

\problem{2b}

For any $(x, y) \in \field^2$ we need $a_1 + 3a_2 = x, 2a_1 + 5a_2 = y$. This
yields the unique solution $(a_1, a_2) = (-5x + 3y, 2x - y)$, so this is a
basis.

\problem{2c}

Any linear combination is $(a_1 + 7a_2, 2a_1 - 5a_2, -4a_1 + 6a_2)$. In order
for this to equal 0 and yet $a_2 \neq 0$, we would have to have $a_1/a_2 = -1/7,
5/2, 3/2$. So these vectors are LI. But we can pick a vector such that the first
two conditions are satisfied and the third is not, for example picking $a_1 =
a_2 = 1$, pick $(8, -3, 0)$ (the last element could be anything but 2). Only
$a_1 = a_2 = 1$ can match both of the first two elements, which will not match
the third. So these vectors do not span $\field^3$.

\problem{2d}

In problem 2b we showed that $(1,2), (3,5)$ spanned $\field^2$, so this list
trivially does as well. However, we can pick $a_1, a_2$ such that $a_1(1,2) +
a_2(3,5) = (4,13)$, then pick $a_3 = -1$, yielding a linear combination with
$a_i$ not all 0 that sums to 0. So these vectors are LD.

\problem{2e}

The only way to write $(x,x,y)$ in terms of these two vectors is $x(1,1,0) +
y(0,0,1)$, since only the coefficient on $(1,1,0)$ can affect the first two
elements and only that on $(0,0,1)$ can affect the last. So they are a basis.

\problem{2f}

We need to find $a_i$ such that for any $(x, y, z)$, $(a_1 + a_2, -a_1, -a_2) =
(-y-z, y, z)$, simply rewriting the constraint as $x = -y - z$. $a_1 = -y$ and
$a_2 = -z$ by the latter two elements, and this does result in the correct value
for the first element: $a_1 + a_2 = -y - z$. Hence, there is one unique
combination to represent each vector in the space, so this is a basis.

\problem{2g}

Any $p(z) \in \mathcal{P}_m(\field), p(z) = a_0 + \ldots + a_mz^m$ can be only
written in terms of $(1, z, \ldots, z^m)$ as $a_0$ multiplying 1, $a_1$
multiplying $z$, etc. because only one of these vectors is able to affect each
term in $p(z)$. Hence we have a unique representation.

\problem{3a}

$(3,1,0,0,0), (0,0,7,1,0), (0,0,0,0,1)$ is a basis. The coefficient $a_1$ would
uniquely determine the first two elements, $a_2$ the next two, and $a_3$ the
last.

\problem{3b}

We can extend this with the standard basis of $\real^5$ - call the basis above
$v_1, v_2, v_3$, and the standard basis $e_i$. $e_1$ is kept. $e_2$ is not,
because $e_2 = v_1 - 3e_1$. $e_3$ is kept. $e_4$ is not, because $e_4 = v_3 -
7e_3$. And $e_5 = v_3$, so it is dropped. The resulting basis is $v_1, v_2, v_3,
e_1, e_3$.

\problem{3c}

$\text{span}(e_1, e_3)$ defines such a subspace. As these two vectors form a
basis of $\real^5$ with basis vectors of $U$, we have $\real^5 = U + W$. $U
\cap W$ is 0, since the elements of W can only have 0 in the second, fourth,
and fifth elements, which is only possible for $(0,0,0,0,0)$ in both subspaces.
So that makes $U \oplus W$ a direct sum.

\problem{4a}

$v_1 = (6,1,0,0,0)$ serves to form unique combinations of the first two
elements. $v_3 = (0,0,1,1,-1)$ and $v_4 = (0,0,0,3,-2)$ serve to form unique
combinations of the remaining three.

\problem{4b}

With similar notation as before, except this time over $\complex$, we can extend
with $e_1 = (1,0,0,0,0), e_2 = (0,1,0,0,0), \ldots$. $e_1$ is kept. $e_2$ is
not, because $e_2 = v_1 - 6e_i$. $e_3$ is kept.

At this point we have a basis since we have five basis vectors. But for
completion's sake, going through the remaining standard basis vectors: $e_4$
would not be kept, because $v_5 - 3v_3 - 3e_3 = e_4$. $e_5$ would not be kept,
because $e_5 = 3e_4 - v_5$, and we already showed $e_4$ to be a linear
combination.

\problem{4c}

$\text{span}(e_1, e_3)$ defines such a subpsace, for similar reasons as in
problem 3c. These vectors define a basis of $\complex^5$ with basis vectors of
$U$, so $\complex^5 = U + W$, and $U$ and $W$ can only intersect at 0, making $U
\oplus W$ a direct sum.

\problem{5}

Let $p_0 = 1, p_1 = z, p_2 = z^3 + z^2, p_3 = z^3 - z^2$. Then no polynomial has
degree 2 (though two have degree 3). Any polynomial with coefficients $a_i$ can
be represented as a linear combination with coefficients $b_i$ as follows: $a_0
= b_0, a_1 = b_1$, then we have $b_2 - b_3 = a_2$ and $b_2 + b_3 = a_3$, which
gives us $b_2 = (a_2 + a_3)/2, b_3 = (a_2 - a_3)/2$ uniquely.

\problem{6}

For any $v \in V$, write its unique representation as $v = \sum a_iv_i$. Then we
can rewrite this as $v = a_1(v_1 + v_2) + (a_2 - a_1)(v_2 + v_3) + (a_3 - a_2 +
a_1)(v_3 + v_4) + (a_4 - a_3 + a_2 - a_1)v_4$, so these vectors span $V$.

To show LI, suppose we had $a_i$ such that $a_1(v_1 + v_2) + a_2(v_2 + v_3) +
a_3(v_3 + v_4) + a_4v_4 = 0$. Rearrange to get $a_1v_1 + (a_1 + a_2)v_2 + (a_2 +
a_3)v_3 + (a_3 + a_4)v_4 = 0$. Because the $v_i$ are a basis, they must be LI,
so $a_1 = a_1 + a_2 = a_2 + a_3 = a_3 + a_4 = 0$, which means that the $a_i$ are
all 0. So these vectors are LI, making them a basis.

\problem{7}

Counterexample: Consider $V = \real^4$. Let $v_1 = (1,0,0,0), v_2 = (0,1,0,0),
v_3 = (0,0,1,1), v_4 = (0,0,1,-1)$. Any $(a,b,c,d) \in V$ can only be written as
$av_1 + bv_2 + ((c+d)/2)v_3 + ((c-d)/2)v_4$, making these vectors a basis.

However, consider the subspace $U = {(a,b,c,0)}$. $v_1 \in U, v_2 \in U, v_3
\notin U, v_4 \notin U$. But $v_1$ and $v_2$ do not form a basis of $U$.
Basically, the ``trick'' was to construct $v_3$ and $v_4$ such that a linear
combination of them like $v_3 + v_4$ would be in $U$ and would in fact form a
basis along with $v_1$ and $v_2$, but to have $v_3$ and $v_4$ not individually
be in $U$.

\problem{8}

For any $v \in V$, there is a unique representation $v = u + w$, where $u \in U,
w \in W$. Each of those in turn has a unique representation in terms of $u_i$
and $w_i$, respectively. Thus, $v$ has a unique representation in terms of the
$u_i$ and $w_i$ together, so the $u_i$ and $w_i$ together are a basis for $V$.

\subsection{Dimension}

\problem{1}

Pick a basis of $U$. By 2.39, this is also a basis of $V$. Thus any $v \in V$
can be represented as a linear combination of vectors in $U$, meaning that $v
\in U$. So $U \subset V$ and $V \subset U$, making $U = V$.

\problem{2}

Any subspace of $\real^2$ can have dimension 0, 1, or 2.

\begin{enumerate}
\item Dimension 0 can only be the subspace $\{0\}$.
\item Dimension 1 is the subspace defined by the span of any individual non-zero
  vector in $\real^2$, so any scalar multiple $a(x,y)$ for some given $(x,y)$,
  which will give a line through the origin.
\item Dimension 2 can only be the full space $\real^2$, as per problem 1.
\end{enumerate}

\problem{3}

Any subspace of $\real^3$ can have dimension 0, 1, 2, or 3.

\begin{enumerate}
\item Dimension 0 can only be the subspace $\{0\}$.
\item Dimension 1 is the subspace defined by the span of any individual non-zero
  vector in $\real^3$, so any scalar multiple $a(x,y,z)$ for some given
  $(x,y,z)$, which will give a line through the origin.
\item Dimension 2 is the subspace defined by the span of any two non-LI vectors.
  Each vector individually defines a line through the origin, and any linear
  combination of those will be the plane defined by those two lines.
\item Dimension 3 can only be the full space $\real^3$, as per problem 1.
\end{enumerate}

\problem{4a}

Take the standard basis of $\mathcal{P}_3(\field)$: $1, z, z^2, z^3$. Then
multiply each term by $(z-6)$. Any polynomial in $\mathcal{P}_4(\field)$ must be
expressible in this form, and these vectors are linearly independent as they all
have different degrees.

\problem{4b}

Adding basis vectors $1, z, z^2, z^3, z^4$: 1 is kept, but the remainder are
not. $z = (z-6) + 6$, $z^2 = z(z-6) + 6z$ and we showed $z$ was already a linear
combination, and so on.

\problem{4c}

$W = \text{span}(1)$ i.e. any constant polynomial. This intersects $U$ only at
0, since all other elements of $U$ have degree > 0. And any polynomial in
$\mathcal{P}_4(\field)$ can be written by first multiplying $z^3(z-6)$ by the
necessary leading coefficient for $z^4$, then multiplying $z^2(z-6)$ by an
amount that sets the necessary leading coefficient for $z^3$ net of the $z^3$
term introduced by the previous basis polynomial, and so on until the constant
term, which can be set arbitrarily by adding the right constant.

\problem{5a}

$p''(z) = 2a_2 + 6a_3z + 12a_4z^2$. Factor out $(z-6)$, and a basis for the
remaining polynomial is $(6, 12z)$, so a basis for $p''(z)$ satisfying these
conditions is $6(z-6), 12z(z-6)$. (We could also have just written $(z-6),
z(z-6)$, but we'll undo these extra coefficients shortly.)

Taking the antiderivative twice, we get vectors $z^3 - 18z^2, z^4 - 12z^3$.
The $z$ and constant terms can be anything without affecting the condition
$p''(6) = 0$, so a basis is $1, z, z^3 - 18z^2, z^4 - 12z^3$.

\problem{5b}

Addiing the basis vectors: $z^2$ is kept, which gets us to 5 vectors. The rest
would not be: $z^3 = (z^3 - 18z^2) + 18z^2$ and $z^4 = (z^4 - 12z^3) + 12z^3$.

\problem{5c}

$W = \text{span}{(z^2)}$. This only intersects $U$ at 0, since the second
derivative of any element $az^2 \in W$ is $2a$ which can only satisfy $U$'s
condition at $a = 0$, or 0. And any polynomial in $\mathcal{P}_4(\real)$ can be
written in terms of $U + W$ in the same way discussed in problem 4c; set $z^4 -
12z^3$ to match the $z^4$ coefficient, then set $z^3$ to match net of the
existing $z^3$ term, then set $z^2$ to whatever is desired to match, and set $z$
and $1$ freely.

\problem{6a}

This condition corresponds to:
\begin{align*}
  2a_1 + 2^2a_2 + 2^3a_3 + 2^4a_4 = 5a_1 + 5^2a_2 + 5^3a_3 + 5^4a_4
\end{align*}

$a_0$ can be any value. We can express any one of the remaining $a_i$ in terms
of the rest; we'll use $a_1$. Note that a degree 1 polynomial cannot meet this
condition.

To cover degree 2, $(5^2-2^2)a_2 + (5-2)a_1 = 0$, or $a_1 = -7a_2$. So one
vector is $z^2 - 7z$. To cover degree 3, likewise, $a_1 = -39a_3$, so another
vector is $z^3 - 39z$. Finally, to cover degree 4, $a_1 = -203a_4$, so the last
vector is $z^4 - 203z$. So our basis is: $1, z^2-7z, z^3-39z, z^4-203z$. These
are linearly independent, and we can set the coefficients for $1, z^2, z^3, z^4$
to any value, which will constrain the coefficient for $z$.

\problem{6b}

We would add $z$ to form a basis for $\mathcal{P}_4(\field)$. 1 is already in
the basis, and once $z$ is added, the higher-degree basis polynomials are linear
combinations of existing basis polynomials and $z$.

\problem{6c}

$\text{span}(z)$ for similar reasoning: only 0 satisfies $p(2) = p(5)$ for $p(z)
= az$, hence we have a direct sum.

\problem{7a}

We can further restrict the space from problem 6a, looking for $a_i$ that
satisfy the condition that the following expression is the same at $z = 2$ or $z
= 5$ (at which they have the same value) and $z = 6$:
\begin{align}
  a_0 + a_1(z^2-7z) + a_2(z^3 - 39z) + a_3(z^4 - 203z)
\end{align}

Substituting in $z$, we have
\begin{align}
  -10a_1 - 70a_2 - 390a_3 = -6a_1 - 18a_2 + 78a_3
\end{align}

Again, $a_0$ can be any value. $a_1$ cannot be the only non-zero of what
remains. Covering the degree 3 case, if only $a_1$ and $a_2$ are nonzero, we
have $4a_1 + 52a_2 = 0$, or $a_1 + 13a_2 = 0$. This gives us $(z^3 - 39z) -
13(z^2 - 7z) = z^3 - 13z^2 + 52z$ as a basis vector.

Covering the degree 4 case, if only $a_1$ and $a_3$ are nonzero, we have $4a_1 +
468a_3 = 0$, or $a_1 + 117a_3 = 0$, which gives us $(z^4 - 203z) - 117(z^2 - 7z)
= z^4 - 117z^2 + 616z$ as a basis vector.

So as a basis we have: $1, z^3 - 13z^2 + 52z, z^4 - 117z^2 + 616z$. (Sanity
check here - each basis vector does equal the same thing for $z = 2, 5, 6$.)

\problem{7b}

Similar process as before: $z$ and $z^2$ are added to make a basis.

\problem{7c}

As before, $\text{span}(z, z^2)$, for similar reasons.

\problem{8a}

The antiderivative of $p(z)$ is $a_0z + a_1z^2/2 + a_2z^3/3 + a_3z^4/4 +
a_4z^5/5$ (plus a constant). So $\int_{-1}^1 p = 0$ means that:
\begin{align*}
  a_0 + a_1/2 + a_2/3 + a_3/4 + a_4/5 & = -a_0 + a_1/2 - a_2/3 + a_3/4 - a_4/5 \\
  a_0 + a_2/3 + a_4/5 & = 0
\end{align*}

The $a_1$ and $a_3$ terms can be set freely.

Find a vector to cover $a_2$ as the largest non-zero non-odd coefficient: $3a_0
+ a_2 = 0$ gives us $1 - 3z^2$. Find a vector to cover $a_4$ as the largest
such: $5a_0 + a_4 = 0$ gives us $1 - 5a^4$. Thus, our basis is $z, 1 - 3z^2,
z^3, 1 - 5z^4$.

To tie this back to the idea of odd and even functions: the odd terms can have
arbitrary coefficients since they always satisfy $\int_{-1}^1 p = 0$, but the
even terms are constrained to be higher or lower by a constant to make the
integral equal to 0.

\problem{8b}

Similar process as before: add 1, and then all remaining standard-basis vectors
are linear combinations of existing ones.

\problem{8c}

As before, $\text{span}(1)$ i.e. constants.

\problem{9}

TODO: return to this one

\problem{10}

If each $p_j$ has degree $j$, then the only way to get a linear combination of
the $p_j$ to equal 0 is to have the $p_m$ coefficient be 0 (no other way to zero
out the $z^m$ term), which forces the $p_{m-1}$ coefficient to be 0 (no other
way to zero out the $z^{m-1}$ term), etc. This makes the $p_j$ linearly
independent, and of length $m+1$, the same length as the standard basis $1, z,
\ldots, z^m$. So by 2.39, the $p_j$ are a basis.

\problem{11}

By 2.43, $\dim(U \cap W) = 0$, which means that $U \cap W = \{0\}$, making $U
\oplus W$ a direct sum.

\problem{12}

Since $U$ and $W$ are subspaces of $\real^9$, $\dim(U+W) \leq 9$. By 2.43, $\dim
(U \cap W) \geq 1$, which means that $U \cap W \neq \{0\}$.

\problem{13}

$\dim(U + W) \leq 6$, so by 2.43, $\dim(U \cap W) \geq 2$. Take a basis $v_1,
v_2$ of $U \cap W$; these two vectors must be LI, so neither is a scalar
multiple of the other.

\problem{14}

By 2.43, $\dim(U_1 + \ldots + U_m) \leq \dim U_1 + \dim (U_2 + \ldots + U_m)$
and you can keep applying it to show the overall inequality.

\problem{15}

Take a basis $v_i$ of $V$, and let $U_i = \text{span}(v_i)$. Every vector in $V$
can be written as a unique linear combination of $a_iv_i$ for some $a_i$. Since
$a_iv_i \in U_i$, this likewise means that every vector in $V$ can be written as
a unique sum of $u_i \in U_i$, making $V = U_1 \oplus \ldots \oplus U_n$.

\problem{16}

By 2.43, $\dim(U_1 \oplus \ldots \oplus U_m) = \dim U_1 + \dim(U_2 \oplus \ldots
\oplus U_m) - \dim(U_1 \cap (U_2 \oplus \ldots \oplus U_m)$. But by 1.45, the
intersection in the last term must be $\{0\}$, which has dimension 0. Repeatedly
applying 2.43 on what remains gives us the desired result.

\problem{17}

Counterexample: let $U_1$, $U_2$, $U_3$ be three lines in $\real^2$ that
intersect only in the origin. Then $\dim(U_1 + U_2 + U_3) = 2$, but the
dimension of any intersection is 0, so the right-hand side is $\dim U_1 + \dim
U_2 + \dim U_3 = 3$.

\end{document}