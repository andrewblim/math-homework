\documentclass{article}

\usepackage{amssymb}
\usepackage{amsmath}

\begin{document}
\title{Sheldon Axler - Linear Algebra Done Right, 3rd edition - Chapter 1}
\author{Andrew Lim}

\def \problem#1{\subsubsection*{Problem #1}}
\def \real{\mathbf{R}}
\def \complex{\mathbf{C}}
\def \field{\mathbf{F}}

Notations are the same as what Axler uses. For example, we use boldface $\real$
for the reals, and $\field$ is $\real$ or $\complex$, not a general field.

\section{Vector spaces}

\subsection{$\real^n$ and $\complex^n$}

\problem{1}

$ac - bd + (ad + bc)i = 1$, so $ac - bd = 1$ and $ad + bc = 0$.

\begin{itemize}
\item If $a = 0$, then $c = 0$, since $b$ cannot also be 0, and $d = -1/b$.
\item Otherwise, solve it like a typical two-equation system. $d = -bc/a$, then
  $ac + b^2c/a = 1$, which gets us $c = a/(a^2+b^2)$, and $d = -b/(a^2+b^2)$.
\end{itemize}

As it turns out, $c = a/(a^2+b^2)$, and $d = -b/(a^2+b^2)$ work as general
expressions when $a = 0$ as well.

\problem{2}

Multiply it all out:

\begin{align*}
  & (1/8)((-1)^3 + 3(-1)^2\sqrt{3}i + 3(-1)(\sqrt{3}i)^2 + (\sqrt{3}i)^3) \\
  & = (1/8)(-1 + 3\sqrt{3}i + 9 - 3\sqrt{3}i) \\
  & = 1 \\
\end{align*}

\problem{3}

We need $(a+bi)^2 = 0+1i$, so $a^2 - b^2 = 0$ and $2ab = 1$. From the first,
$a = \pm b$. Then from the second, either $a = b = 1/\sqrt{2}$, or $a = b =
-1/\sqrt{2}$. So our two roots of $i$ are $(1 + i)/\sqrt{2}$ and $(-1 -
i)/\sqrt{2}$.

\problem{4}

Say $\alpha = a_1 + a_2i$ and $\beta = b_1 + b_2i$. Then

\begin{align*}
  \alpha + \beta & = a_1 + b_1 + (a_2 + b_2)i \\
                 & = b_1 + a_1 + (b_2 + a_2)i = \beta + \alpha
\end{align*}

\problem{5}

Notation as before, but also say $\lambda = c_1 + c_2i$. Then

\begin{align*}
  (\alpha + \beta) + \lambda & = (a_1 + b_1) + c_1 + ((a_2 + b_2) + c_2)i \\
                             & = a_1 + (b_1 + c_1) + (a_2 + (b_2 + c_2))i = \alpha + (\beta + \lambda)
\end{align*}

\problem{6}

Notation as before.

\begin{align*}
  (\alpha\beta) & = (a_1b_1 - a_2b_2) + (a_1b_2 + a_2b_1)i \\
  (\beta\lambda) & = (b_1c_1 - b_2c_2) + (b_1c_2 + b_2c_1)i \\
  (\alpha\beta)\lambda & = (a_1b_1 - a_2b_2)c_1 - (a_1b_2 + a_2b_1)c_2 \\
                & \quad + ((a_1b_1 - a_2b_2)c_2 + (a_1b_2 + a_2b_1)c_1)i \\
                & = (a_1(b_1c_1 - b_2c_2) - a_2(b_1c_2 + b_2c_1)) \\
                & \quad + (a_1(b_1c_2 + b_2c_1) + a_2(b_1c_1 - b_2c_2))i \\
                & = \alpha(\beta\lambda)
\end{align*}

\problem{7}

If $\alpha = a_1 + a_2i$, then $\beta = -a_1 - a_2i$ satisfies $\alpha + \beta =
0$. If there were two $\beta_1, \beta_2$ satisfying this, then $\alpha + \beta_1
= \alpha + \beta_2$, and then adding $\beta$ as defined above to both sides
would yield $\beta_1 = \beta_2$, hence $\beta$ is unique.

\problem{8}

In problem 1 we showed that an inverse of $\alpha$ exists. Now suppose there are
two numbers $\beta_1, \beta_2$ satisfying $\alpha\beta_1 = 1$ and $\alpha\beta_2
= 1$. Then $\alpha\beta_1\beta_2 = \beta_2$, and using associativity (problem 6)
and commutativity (1.4 in the book), we can get $\beta_1 = \beta_2$.

\problem{9}

Notation as before.

\begin{align*}
\lambda(\alpha + \beta) & = (c_1 + c_2i)(a_1 + b_1 + (a_2 + b_2)i) \\
                        & = c_1(a_1 + b_1) - c_2(a_2 + b_2) + (c_1(a_2 + b_2) + c_2(a_1 + b_1))i \\
                        & = c_1a_1 - c_2a_2 + (c_1a_2 + c_2a_1)i + c_1b_1 - c_2b_2 + (c_1b_2 + c_2b_1)i \\
                        & = (c_1 + c_2i)(a_1 + a_2i) + (c_1 + c_2i)(b_1 + b_2i) = \lambda\alpha + \lambda\beta
\end{align*}

\problem{10}

Just algebra, $x = (1/2, 6, -7/2, 1/2)$.

\problem{11}

Defining $\lambda = a + bi$, to satisfy the first element, we need $(a+bi)(2-3i)
= 12-5i$. That already constrains us to $\lambda = 3 + 2i$. This also works for
the second element: $(3 + 2i)(5 + 4i) = 7 + 22i$. But it does not work for the
third element.

\problem{12}

\begin{align*}
(x + y) + z & = (x_1 + y_1, \ldots, x_n + y_n) + (z_1, \ldots, z_n) \\
            & = (x_1 + y_1 + z_1, \ldots, x_n + y_n + z_n) \\
            & = (x_1 + (y_1 + z_1), \ldots, x_n + (y_n + z_n)) \\
            & = (x_1, \ldots, x_n) + (y_1 + z_1, \ldots, y_n + z_n) = x + (y + z)
\end{align*}

\problem{13}

\begin{align*}
(ab)x & = (abx_1, \ldots, abx_n) = a(bx_1, \ldots, bx_n) = a(bx)
\end{align*}

\problem{14}

\begin{align*}
1x & = (1x_1, \ldots, 1x_n) = (x_1, \ldots, x_n) = x
\end{align*}

\problem{15}

\begin{align*}
\lambda(x+y) & = \lambda(x_1 + y_1, \ldots, x_n + y_n) \\
             & = \lambda(x_1, \ldots, x_n) + \lambda(y_1, \ldots, y_n) = \lambda x + \lambda y
\end{align*}

\problem{16}

\begin{align*}
(a+b)x & = ((a+b)x_1, \ldots, (a+b)x_n) \\
       & = (ax_1 + bx_1, \ldots, ax_n + bx_n) \\
       & = (ax_1, \ldots, ax_n) + (bx_1, \ldots, bx_n) = ax + bx
\end{align*}

\subsection{Definition of Vector Space}

\problem{1}

By definition, $-(-v) + (-v) = 0$. Add $v$ to both sides: $-(-v) = v$.

\problem{2}

If $a \neq 0$, then it has a multiplicative inverse $1/a$. So take $av = 0$
and multiply both sides by $1/a$ to get $v = 0$. Hence, either $a \neq 0$,
forcing $v = 0$, or $a = 0$.

\problem{3}

Given 1.26, that every element in $V$ has a unique additive inverse, $-v$ is
uniquely defined, so $w - v$ and $(w - v)/3$ are also uniquely defined, and $x =
(w - v)/3$ is as well.

\problem{4}

The empty set has no additive identity, and thus is not a vector space.

\problem{5}

The fact that the vector space definitions implies $0v = 0$ was proved in 1.29.
We need to show that the vector space definitions without the additive identity
but with $0v = 0$ implies the additive identity.

$0v = 0$ implies $(1 + (-1))v = 0$, where 1 is the multiplicative identity in
$\field$ and -1, which must exist, is its inverse. Then distribute, $1v + (-1)v =
0$, and then simplify $v + (-1)v = 0$. This implies that $(-1)v$ is the additive
inverse of $v$.

\problem{6}

This is not a vector space, as it fails the distributive property. $(2 -
1)(\infty) = 1(\infty) = \infty$ when you do not distribute, and $(2 -
1)(\infty) = 2(\infty) + (-1)\infty = \infty + (-\infty) = 0$ when you do
distribute.

\end{document}