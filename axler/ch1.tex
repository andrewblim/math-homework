\documentclass{article}

\usepackage{amssymb}
\usepackage{amsmath}
\usepackage[parfill]{parskip}

\begin{document}
\title{Sheldon Axler - Linear Algebra Done Right, 3rd edition - Chapter 1}
\author{Andrew Lim}

\def \problem#1{\subsubsection*{Problem #1}}
\def \real{\mathbf{R}}
\def \complex{\mathbf{C}}
\def \field{\mathbf{F}}

Notations are the same as what Axler uses. For example, we use boldface $\real$
for the reals, and $\field$ is $\real$ or $\complex$, not a general field.

\section{Vector spaces}

\subsection{$\real^n$ and $\complex^n$}

\problem{1}

$ac - bd + (ad + bc)i = 1$, so $ac - bd = 1$ and $ad + bc = 0$.

\begin{itemize}
\item If $a = 0$, then $c = 0$, since $b$ cannot also be 0, and $d = -1/b$.
\item Otherwise, solve it like a typical two-equation system. $d = -bc/a$, then
  $ac + b^2c/a = 1$, which gets us $c = a/(a^2+b^2)$, and $d = -b/(a^2+b^2)$.
\end{itemize}

As it turns out, $c = a/(a^2+b^2)$, and $d = -b/(a^2+b^2)$ work as general
expressions when $a = 0$ as well.

\problem{2}

Multiply it all out:

\begin{align*}
  & (1/8)((-1)^3 + 3(-1)^2\sqrt{3}i + 3(-1)(\sqrt{3}i)^2 + (\sqrt{3}i)^3) \\
  & = (1/8)(-1 + 3\sqrt{3}i + 9 - 3\sqrt{3}i) \\
  & = 1 \\
\end{align*}

\problem{3}

We need $(a+bi)^2 = 0+1i$, so $a^2 - b^2 = 0$ and $2ab = 1$. From the first,
$a = \pm b$. Then from the second, either $a = b = 1/\sqrt{2}$, or $a = b =
-1/\sqrt{2}$. So our two roots of $i$ are $(1 + i)/\sqrt{2}$ and $(-1 -
i)/\sqrt{2}$.

\problem{4}

Say $\alpha = a_1 + a_2i$ and $\beta = b_1 + b_2i$. Then

\begin{align*}
  \alpha + \beta & = a_1 + b_1 + (a_2 + b_2)i \\
                 & = b_1 + a_1 + (b_2 + a_2)i = \beta + \alpha
\end{align*}

\problem{5}

Notation as before, but also say $\lambda = c_1 + c_2i$. Then

\begin{align*}
  (\alpha + \beta) + \lambda & = (a_1 + b_1) + c_1 + ((a_2 + b_2) + c_2)i \\
                             & = a_1 + (b_1 + c_1) + (a_2 + (b_2 + c_2))i = \alpha + (\beta + \lambda)
\end{align*}

\problem{6}

Notation as before.

\begin{align*}
  (\alpha\beta) & = (a_1b_1 - a_2b_2) + (a_1b_2 + a_2b_1)i \\
  (\beta\lambda) & = (b_1c_1 - b_2c_2) + (b_1c_2 + b_2c_1)i \\
  (\alpha\beta)\lambda & = (a_1b_1 - a_2b_2)c_1 - (a_1b_2 + a_2b_1)c_2 \\
                & \quad + ((a_1b_1 - a_2b_2)c_2 + (a_1b_2 + a_2b_1)c_1)i \\
                & = (a_1(b_1c_1 - b_2c_2) - a_2(b_1c_2 + b_2c_1)) \\
                & \quad + (a_1(b_1c_2 + b_2c_1) + a_2(b_1c_1 - b_2c_2))i \\
                & = \alpha(\beta\lambda)
\end{align*}

\problem{7}

If $\alpha = a_1 + a_2i$, then $\beta = -a_1 - a_2i$ satisfies $\alpha + \beta =
0$. If there were two $\beta_1, \beta_2$ satisfying this, then $\alpha + \beta_1
= \alpha + \beta_2$, and then adding $\beta$ as defined above to both sides
would yield $\beta_1 = \beta_2$, hence $\beta$ is unique.

\problem{8}

In problem 1 we showed that an inverse of $\alpha$ exists. Now suppose there are
two numbers $\beta_1, \beta_2$ satisfying $\alpha\beta_1 = 1$ and $\alpha\beta_2
= 1$. Then $\alpha\beta_1\beta_2 = \beta_2$, and using associativity (problem 6)
and commutativity (1.4 in the book), we can get $\beta_1 = \beta_2$.

\problem{9}

Notation as before.

\begin{align*}
\lambda(\alpha + \beta) & = (c_1 + c_2i)(a_1 + b_1 + (a_2 + b_2)i) \\
                        & = c_1(a_1 + b_1) - c_2(a_2 + b_2) + (c_1(a_2 + b_2) + c_2(a_1 + b_1))i \\
                        & = c_1a_1 - c_2a_2 + (c_1a_2 + c_2a_1)i + c_1b_1 - c_2b_2 + (c_1b_2 + c_2b_1)i \\
                        & = (c_1 + c_2i)(a_1 + a_2i) + (c_1 + c_2i)(b_1 + b_2i) = \lambda\alpha + \lambda\beta
\end{align*}

\problem{10}

Just algebra, $x = (1/2, 6, -7/2, 1/2)$.

\problem{11}

Defining $\lambda = a + bi$, to satisfy the first element, we need $(a+bi)(2-3i)
= 12-5i$. That already constrains us to $\lambda = 3 + 2i$. This also works for
the second element: $(3 + 2i)(5 + 4i) = 7 + 22i$. But it does not work for the
third element.

\problem{12}

\begin{align*}
(x + y) + z & = (x_1 + y_1, \ldots, x_n + y_n) + (z_1, \ldots, z_n) \\
            & = (x_1 + y_1 + z_1, \ldots, x_n + y_n + z_n) \\
            & = (x_1 + (y_1 + z_1), \ldots, x_n + (y_n + z_n)) \\
            & = (x_1, \ldots, x_n) + (y_1 + z_1, \ldots, y_n + z_n) = x + (y + z)
\end{align*}

\problem{13}

\begin{align*}
(ab)x & = (abx_1, \ldots, abx_n) = a(bx_1, \ldots, bx_n) = a(bx)
\end{align*}

\problem{14}

\begin{align*}
1x & = (1x_1, \ldots, 1x_n) = (x_1, \ldots, x_n) = x
\end{align*}

\problem{15}

\begin{align*}
\lambda(x+y) & = \lambda(x_1 + y_1, \ldots, x_n + y_n) \\
             & = \lambda(x_1, \ldots, x_n) + \lambda(y_1, \ldots, y_n) = \lambda x + \lambda y
\end{align*}

\problem{16}

\begin{align*}
(a+b)x & = ((a+b)x_1, \ldots, (a+b)x_n) \\
       & = (ax_1 + bx_1, \ldots, ax_n + bx_n) \\
       & = (ax_1, \ldots, ax_n) + (bx_1, \ldots, bx_n) = ax + bx
\end{align*}

\subsection{Definition of Vector Space}

\problem{1}

By definition, $-(-v) + (-v) = 0$. Add $v$ to both sides: $-(-v) = v$.

\problem{2}

If $a \neq 0$, then it has a multiplicative inverse $1/a$. So take $av = 0$
and multiply both sides by $1/a$ to get $v = 0$. Hence, either $a \neq 0$,
forcing $v = 0$, or $a = 0$.

\problem{3}

Given 1.26, that every element in $V$ has a unique additive inverse, $-v$ is
uniquely defined, so $w - v$ and $(w - v)/3$ are also uniquely defined, and $x =
(w - v)/3$ is as well.

\problem{4}

The empty set has no additive identity, and thus is not a vector space.

\problem{5}

The fact that the vector space definitions implies $0v = 0$ was proved in 1.29.
We need to show that the vector space definitions without the additive identity
but with $0v = 0$ implies the additive identity.

$0v = 0$ implies $(1 + (-1))v = 0$, where 1 is the multiplicative identity in
$\field$ and -1, which must exist, is its inverse. Then distribute, $1v + (-1)v =
0$, and then simplify $v + (-1)v = 0$. This implies that $(-1)v$ is the additive
inverse of $v$.

\problem{6}

This is not a vector space, as it fails the distributive property. $(2 -
1)(\infty) = 1(\infty) = \infty$ when you do not distribute, and $(2 -
1)(\infty) = 2(\infty) + (-1)\infty = \infty + (-\infty) = 0$ when you do
distribute.

\subsection{Subspaces}

\problem{1a}

$(0,0,0)$ is in this set. For any $x, y$, add $(x_1 + y_1) + 2(x_2 + y_2) +
3(x_3 + y_3) = 0$, so $x+y$ is also in this set. And for any $a \in \field$,
$ax_1 + 2ax_2 + 3ax_3 = 0$, so $ax$ is also in this set. Thus, this is a
subspace.

\problem{1b}

This does not include $(0,0,0)$.

\problem{1c}

This is not closed under addition. $(1,1,0)$ and $(0,0,1)$ are both in this set,
but their sum is not.

\problem{1d}

This is a subspace, following a very similar approach to problem 1a.

\problem{2a}

$\Rightarrow$: in order for it to be a subspace, 0 must be in it, so
substituting into the restriction we get $b = 0$. $\Leftarrow$: if $b = 0$, we
can follow the same steps straightforwardly as in problem 1a to show that this
is a subspace.

\problem{2b}

The additive identity $0(x) = 0$ is continuous, so it is in this set. The sum of
two continuous functions is continuous, so it is closed under addition. A scalar
multiple of a continuous function is continuous, so it is closed under scalar
multiplication. Hence, this is a subspace of $\real^{[0, 1]}$.

\problem{2c}

The additive identity $0(x) = 0$ is differentiable. The sum of two
differentiable functions is differentiable, as is a scalar multiple of a
differential function. Hence, this is a subspace of $\real^\real$.

\problem{2d}

$\Rightarrow$: The set must include $0(x)$, and the derivative of that will be 0
everywhere as well, so $b = 0$. $\Leftarrow$: If $b = 0$, $0(x)$ is included,
the sum of any two functions is included ($(f+g)'(2) = f'(2) + g'(2) = 0$), and
a scalar multiple of any function is included ($(af)'(2) = af'(2) = 0$). Hence,
this is a subspace.

\problem{2e}

The 0 sequence $(0, 0, \ldots)$ has limit 0, so is in this set. The limit of the
sum of any two sequences in $\complex$ is the sum of their limits, which is 0,
so this set is closed under addition. The limit of a scalar multiple of any
sequence in $\complex$ is the scalar multiple of the limit, which is 0, so this
set is closed under scalar multiplication. Hence, this is a subspace.

\problem{3}

\begin{itemize}
\item It contains $0(x)$, since its derivatives are all 0.
\item Suppose we have two functions $f$ and $g$ in the set. Then $(f+g)'(1) =
  f'(1) + g'(1) = 3f(2) + 3g(2) = 3(f+g)(2)$. So it is closed under addition.
\item Suppose we have $f$ and a scalar $a$. Then $(af)'(1) = af'(1) = 3af(2) =
  3(af)(2)$. So it is closed under scalar multiplication.
\end{itemize}

\problem{4}

$\Rightarrow$: $0(x)$ must be in it, so $b = 0$. $\Leftarrow$: If $b = 0$, then
$0(x)$ is in the set, any $f$ and $g$ have $\int_0^1 (f+g) = \int_0^1 f +
\int_0^1 g = 0$, and $\int_0^1 af = a \int_0^1 f = 0$.

\problem{5}

$\real^2$ is not closed under scalar multiplication, since here we are
considering scalars from $\complex$. So it is not a subspace of $\complex^2$.

\problem{6a}

It's pretty clear that 0 is included and that any scalar multiple of an element
in the set will still be in the set. Mainly we need to look at addition.

For $a = (a_1, a_2, a_3)$ and $b = (b_1, b_2, b_3)$ in the set, we want to see
if $(a_1 + b_1)^3 = (a_2 + b_2)^3$. We know that $a_1^3 = a_2^3$ and $b_1^3 =
b_2^3$, so if we multiply both sides out and divide by 3, we are left with the
question: does $a_1^2b_1 + a_1b_1^2 = a_2^2b_2 + a_2b_2^2$?

Any complex number has only one real cubic root, so under $\real$, we know that
$a_1 = a_2$ and $b_1 = b_2$. If we make this substitution above, we see that it
holds, so this set is a subspace of $\real^3$.

\problem{6b}

Same setup as before. However, it is possible for $a_1 \neq a_2$ and $b_1 \neq
b_2$, since two different numbers can cube to the same number. For example, both
1 and $-1/2 + (\sqrt{3}/2)i$ cube to 1 (problem 2 from 1A). Hence, $(1, 1, 0)$ and
$(1, -1/2 + (\sqrt{3}/2)i, 0)$ are both in the set, but their sum $(2, 1/2 +
(\sqrt{3}/2)i, 0)$ is not ($8 \neq -1$). So this is not a subspace of
$\complex^3$.

\problem{7}

The set $\{(x, 0) : x \in \mathbf{Z}\}$ is closed under addition and additive
inverse, but is not closed under scalar multiplication in $\real$, simply by
picking a non-integer scalar.

\problem{8}

The ``coordinate axes'' set $\{(x, y) : x = 0 \text{ or } y = 0\}$ is closed
under scalar multiplication, but not addition.

\problem{9}

$0(x)$ is trivially periodic. For any two periodic functions $f$ and $g$ with
periods $p_f$ and $p_g$, their sum has a period of the least common multiple of
$p_f$ and $p_g$. Any scalar multiple of $f$ has the same period as before, since
$af(x) = af(x + p)$. Hence, this is a subspace.

\problem{10}

$U_1 \cap U_2$ includes 0, since they both individually do. For any $v, w \in
U_1 \cap U_2$, $v + w \in U_1$ and $v + w \in U_2$ separately, so $v + w \in U_1
\cap U_2$ as well. Finally, $av \in U_1$ and $av \in U_2$ separately, so $av \in
U_1 \cap U_2$. Hence, this is a subspace.

\problem{11}

The previous argument can just be extended to any number of subspaces $U_i$; the
intersection must contain 0, any two vectors' sum is in each element of the
collection and hence in the intersection, and any scaled vector is in each
element of the collection and hence in the intersection.

\problem{12}

$\Rightarrow$: Call these subspaces $U_1, U_2$. Suppose we were able to pick $v,
w \in U_1 \cup U_2$ such that $v \in U_1, v \notin U_2$ and $w \in U_2, w \notin
U_1$. By closure of addition, $v + w$ must be in either $U_1$ or $U_2$ (or
both). Without loss of generality, let's say $v + w \in U_1$. Then $v + w + (-v)
\in U_1$, since $U_1$ is itself a subspace and is closed under additive inverse.
This gets us $w \in U_1$, which is a contradiction. Thus, it is not possible to
pick two such vectors; all vectors in at least one of the two subspaces must
also be in the other. In other words, one subspace is contained in the other.

$\Leftarrow$: If one subspace is contained in the other, their union is simply
the larger subspace, which is a subspace, trivially.

\problem{13}

(TODO. Have to come back to this one!)

\problem{14}

Any sum of $u \in U, w \in W$ has the same first two elements, so is in $U + W$.
Any element in $U + W$, call it $(x, x, y, z)$, can be broken down into $(x, x,
y, y) \in U$ and $(0, 0, 0, z-y) \in W$. Hence $U + W$ is equivalent to the set
in question.

\problem{15}

Subspaces are closed under addition, so the sum of any $u_1, u_2 \in U$ is also
in $U$. Likewise, any element $u_1 + u_2 \in U + U$ must be in $U$, since $u_1$
and $u_2$ are individually in $U$. So $U + U = U$.

\problem{16}

Sums are commutative. Any $u + w \in U + W$ is equal to $w + u$, which is in $W
+ U$, and vice versa.

\problem{17}

Sums are associative. Any $(u_1 + u_2) + u_3 \in (U_1 + U_2) + U_3$ is equal to
$u_1 + (u_2 + u_3)$, which is in $U_1 + (U_2 + U_3)$, and vice versa.

\problem{18}

The set $\{0\}$ is an additive identity, and the only one, since $\{0\}$ itself
is a subspace and nothing but itself can be added to itself and still give us
$\{0\}$.

Sums of subspaces contain their individual subspaces, since 0 must be in every
subspace and adding 0 to any element yields itself. Thus, the only subspace that
can have an additive identity is $\{0\}$, since no other subspace can have
another subspace added to it and yield a single-element subspace.

\problem{19}

Counterexample: $U_1$ is some subspace of $W$ smaller than $W$ itself, and $U_2
= W$. Then $U_1 + W = U_2 + W = W$, but $U_1 \neq U_2$. (To make this more
genuinely concrete, say that for $V = \real^2$, $U_1$ is the x-axis, and $U_2$
and $W$ are $V$ itself.)

\problem{20}

Subspace: $W = \{(0, x, 0, y) \in \field^4 : x, y \in \field \}$

This clearly satisfies the subspace conditions, and the only place it intersects
$U$ is where $x = 0$ and $y = 0$ in the definition of $U$, which only happens at
$(0,0,0,0)$. So $U \cap W = \{0\}$, making $U \oplus W$ a direct sum.

Any $(x_1, x_2, x_3, x_4) \in \field^4$ can be expressed as $(x_1, x_1, x_3,
x_3) + (0, x_2-x_1, 0, x_4-x_3)$, so the direct sum does equal $\field^4$ itself.

\problem{21}

Subspace: $W = \{(0, 0, x, y, z) \in \field^5 : x, y, z \in \field \}$

This clearly satisfies the subspace conditions, and the only place it intersects
$U$ is where $x = 0$ and $y = 0$ in the definition of $U$, which only happens at
$(0,0,0,0,0)$. So $U \cap W = \{0\}$, making $U \oplus W$ a direct sum.

Any $(x_1, x_2, x_3, x_4, x_5) \in \field^5$ can be expressed as $(x_1, x_2, x_1
+ x_2, x_1 - x_2, 2x_1) + (0, 0, x_3 - x_2 - x_1, x_4 + x_2 - x_1, x_5 - 2x_1)$,
so the direct sum does equal $\field^4$ itself.

\problem{22}

Subspaces: $W_1 = \{(0,0,x,0,0)\}$, $W_2 = \{(0,0,0,y,0)\}$, and $W_3 =
\{(0,0,0,0,z)\}$. These are clearly individually subspaces, their sum is
$\{(0,0,x,y,z)\}$, which is clearly also a subspace, and their sum is direct,
since they only intersect at 0. We already showed associativity of subspace sums
in problem 17, so we just need to show that $U + (W_1 \oplus W_2 \oplus W_3)$ is
a direct sum. But these two only intersect at 0 as well. Hence, $U \oplus W_1
\oplus W_2 \oplus W_3$ is a direct sum.

Any $(x_1, x_2, x_3, x_4, x_5) \in \field^5$ can be expressed as $(x_1, x_2, x_1
+ x_2, x_1 - x_2, 2x_1) + (0, 0, x_3 - x_2 - x_1, 0, 0) + (0, 0, 0, x_4 + x_2 -
x_1, 0) + (0, 0, 0, 0, x_5 - 2x_1)$, so the direct sum does equal $\field^5$ itself.

\problem{23}

Counterexample: $W = \{(x,0)\}, U_1 = \{(0,x)\}, U_2 = \{(x,x)\}$ over $V =
\real^2$. Each of these is individually a subspace: they contain 0, they are
closed under addition, and they are closed under scalar multiplication. $W \cap
U_1$ and $W \cap U_2$ are both $\{(0,0)\}$, so $U_1 \oplus W$ and $U_2 \oplus W$
are both direct sums. Finally, any $(x,y) \in \real^2$ can be written as $(x,0)
+ (y,0)$ or $(x-y,0) + (y,y)$, so the direct sums both cover all of $\real^2$.
But, of course, $U_1 \neq U_2$.

\problem{24}

The even functions are a subspace, since $0(x)$ is even, $(f+g)(-x) = f(-x) +
g(-x) = f(x) + g(x) = (f+g)(x)$, and $(af)(-x) = af(-x) = af(x) = (af)(x)$.

The odd functions are a subspace, since $0(x)$ is odd, $(f+g)(-x) = f(-x) +
g(-x) = -f(x) - g(x) = -(f(x) + g(x)) = -(f+g)(x)$, and $(af)(-x) = af(-x) =
-af(x) = -(af)(x)$.

The evens and odds only intersect at $0(x)$, because for a function $f$ to be in
both, $f(x) = -f(x)$ everywhere, which is only true of $0(x)$. Thus, the evens
and the odds directly sum to $\real^\real$.

\end{document}